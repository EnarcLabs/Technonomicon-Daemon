\documentclass[12pt]{article}

\title{Technonomicon API Documentation}
\author{Olivia Trewin}
\date{\today\\ \normalsize Version 0.1}

\begin{document}

    \maketitle

    \section{Introduction}
        The Technonomicon is a reverse-engineering grimoire.
        It allows the public to view and create records of reverse engineering efforts and results.
        These records are organized to make browsing easier and facilitate teamwork on complex reverse-engineering problems.

        \subsection{High-level Overview}
            Interface with the Technonomicon is done using a REST API, which uses JWT tokens for authentication.
            Users are identified with a username, use their email for password resets, and have a user-specified password which can be used to receive an API token.
            All interactions with the API are stateless, in keeping with REST practices.
        
        \subsection{Purpose of this Document}
            This document intends to describe every possible interaction with the REST API, as well as provide some useful example pseudocode.
            To see usage information for reference bindings for various languages, consult the binding repository or, if present, the package's man page.

    \section{User Creation and Authentication}
        These calls are for creating users and retrieving authentication tokens.

        \subsection{/api/register}
            \begin{itemize}
                \item HTTP Verb: POST
                \item Authorization Required: No
                \item Parameters:
                \begin{itemize}
                    \item username - String - The username of the user to register.
                    \item email - String - The email of the user to register.
                    \item password - String - The password of the user to register.
                \end{itemize}
                \item Response:
                \begin{itemize}
                    \item success - Boolean - True if registration succeeded, false otherwise.
                    \item errors - String[] - A list of errors that caused the registration to fail. Undefined if the registration succeeded.
                \end{itemize}
            \end{itemize}
            \subsubsection{Description}
                Attempts to register a new user with the specified username, email, and password. If this succeeds, it will send a validation email to the specified address.
                This should be followed with a call to /api/validate once the end user receives the validation token.
            
        \subsection{/api/validate}
            \begin{itemize}
                \item HTTP Verb: POST
                \item Authorization Required: No
                \item Parameters:
                \begin{itemize}
                    \item username - String - The username of the user to register.
                    \item emailToken - String - The validation token the user received in their email.
                \end{itemize}
                \item Response:
                \begin{itemize}
                    \item success - Boolean - True if registration succeeded, false otherwise.
                    \item errors - String[] - A list of errors that caused the registration to fail. Undefined if the registration succeeded.
                \end{itemize}
            \end{itemize}
            \subsubsection{Description}
                Attempts to validate the user's email address. After this call succeeds, the user is allowed to receive API tokens and call further functions.
    
        \subsection{/api/token}
            \begin{itemize}
                \item HTTP Verb: POST
                \item Authorization Required: No
                \item Parameters:
                \begin{itemize}
                    \item username - String - The username of the user to get a token for.
                    \item password - String - The user's password.
                \end{itemize}
                \item Response:
                \begin{itemize}
                    \item On success: A valid JWT token.
                    \item On failure: A 400 Bad Request response.
                \end{itemize}
            \end{itemize}
            \subsubsection{Description}
                Authenticates a user and produces a JWT token that can be used to validate subsequent API calls.
                To use this token on functions which require authorization, add an HTTP header called ``Authorization'' with the value of ``Bearer'', a space, and the received JWT token to all requests.
                By default, this token is valid for 1 day, after which the caller must call /api/token again to receive a new token.
    \section{Data Model Manipulation}
        These calls allow the retrieval and alteration of data model objects, like users, posts, tags, and events.

        \subsection{/api/user}
            \begin{itemize}
                \item HTTP Verb: GET
                \item Authorization Required: No
                \item Parameters:
                \begin{itemize}
                    \item id - GUID - The unique ID of the user to get information for.
                \end{itemize}
                \item Response:
                \begin{itemize}
                    \item userId - GUID - The unique ID of the specified user.
                    \item username - String - The specified user's username.
                    \item email - String - The specified user's email.
                \end{itemize}
            \end{itemize}
            \subsubsection{Description}
                Retrieve information about a user using a GUID, including their username and email.
\end{document}